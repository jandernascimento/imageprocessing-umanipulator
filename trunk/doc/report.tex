\documentclass{report}

\usepackage{listings}
\usepackage{color}

\begin{document}

\title{Image manipulation - in theoretical approach}

\author{Hamid Mirisaee,
\and Jander Nascimento, 
\and Raquel Oliveira}

\maketitle

\begin{abstract}

We should write our abstract here..

\end{abstract}

\tableofcontents

\vfill

\section{Image type}

	\subsection{PNG}

		\textit{Fill me}.

	\subsection{JPG}

		\textit{Fill me}.

	\subsection{GIF}

		\textit{Fill me}.

	\subsection{PNM}

		\textit{Fill me}.

\section{Filters}

	\subsection{Fundamental}

		\textit{Fill me}.

		\subsubsection{Convolution}

			\textit{Fill me}.

		\subsubsection{Kernel}

			\textit{Fill me}.

	\subsection{Used filters}
		
		\subsubsection{Gradient}

			\textit{Fill me}.
		
		\subsubsection{Mean}

			\textit{Fill me}.

		\subsubsection{Gaussian}

			\textit{Fill me}.

		\subsubsection{Laplacian}

			\textit{Fill me}.

\section{Color space}          

	\subsection{RGB}
		
		\textit{Fill me}.

	\subsection{CYMK}

		\textit{Fill me}.

	\subsection{Gray scale conversion}

		\textit{Fill me}.

\section{Cropping}

	\textit{Fill me}.

\section{Fusion}

	\textit{Fill me}.

\section{Resizing}

	Image resizing consist in convert an image to a size in which may or not respect the previous ratio.
	When dealing with reducing the size of an image is quite simple to do it, but when dealing with enlarging 
	the image, the process is a little bit complex.

	Reducing the dimension of an image consist in removing as many lines and as many columns as necessary to reach target dimension, of course this
	process may create some side-effect in the image, 

	When we stretch an image, we have few known pixels (dots which are image composition, think like the cells of a human being).

	There is a lot of algorithms that helps to guess (fill the gaps) those unknown pixels. for instance:

	\begin{itemize}
	  \item Nearest neighbor
	  \item Linear interpolation
	  \item Bilinear interpolation
	  \item Bicubic interpolation
	\end{itemize}

	The adopted algorithm is, \textbf{Bilinear interpolation},  due its speed and simplicity.
	
	\subsection{Bilinear interpolation}

	To better understand the bilinear interpolation we must take a look at linear interpolation.
	
	

\section{Histogram}

	\subsection{Color histogram}

		\textit{Fill me}.

	\subsection{Histogram stretching}

		\textit{Fill me}.

\end{document}
